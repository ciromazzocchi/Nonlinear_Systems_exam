\section{Introduction}
The use of model based on epidemiological epidemiological date back to XVIII century, when Daniel Bernoulli perform a study that aims to verify efficiency of vaccines against variola.

Two cenutry later, constant growth of interest in biomathematics cause evolution this subject. Indeed, formulation of differential epidemiological model, called SIR, is due to Karmack and McKendrick \cite{bib:kermack_mckendrick}. This model is focused on cases of epidemic that, after recovery of infected people, it leaves they immune from a new contagion.

In this report has been reported study of a nonlinear epidemiological model SIR to evalute evolution of an infection in a constat population without a vaccine.

Section 2 as been dedicated to description of epidemiological model, defining assumtpion and descriving model's parameters that influences system.

Section 3 is dedidcated to study of epidemiological model, studying properties of all possible invariat set that partecipates to evolution of model, their stability and analyzing all possible bifurcation of codimension 1. It will be shown that it doesn't exist periodic solution in region of interest and that changing a parameter equilbria changes own stability.
Moreover, in this section has been reported simulations of epidemiological model with different values of parameters.

Section 4 is dedicated to an advanced study of model, taking into account possibility that a national government applies a national lockdown to limits number of contagion. This possibilities has been modelled using a piecewise smooth system.

In the end, section 5 is dedicated to conclusion and future development of this report.

All the figures has been generated with MATLAB and Simulink. Scripts has been upload to following Github's repository: \url{https://github.com/ciromazzocchi/Nonlinear_Systems_exam}