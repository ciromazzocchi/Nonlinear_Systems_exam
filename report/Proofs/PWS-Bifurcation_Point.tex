\begin{theorem} Suppose that $R_0 = 1+\delta\frac{\beta}{\mu}$. $(\textbf{x}^*,\gamma^*) = \left(\frac{\mu}{\mu+\delta\beta},\delta,\frac{\mu\beta}{\mu+\delta\beta}-\mu\right)$ is a boundary equilibrium bifurcation point for System (\ref{eq:sir_model_pws_1}).
\end{theorem}

\begin{proof}
First of all, is important to notice that
 
\begin{equation}
    R_0^* = \frac{\beta}{\mu+\gamma^*} = 1+\delta\frac{\beta}{\mu} \implies \gamma^* = \frac{\mu\beta}{\mu+\beta\delta} - \mu
\end{equation}

In $(\textbf{x}^*,\gamma^*) = \left(\frac{\mu}{\mu+\delta\beta},\delta,\frac{\mu\beta}{\mu+\delta\beta}-\mu\right)$ it happens that 

\begin{equation}
\label{eq:bifurcation_point_cond_1}
    \begin{array}{ccccc}
        f^-(\textbf{x}^*,\gamma^*) &=& \begin{pmatrix} 0 \\ 0 \end{pmatrix} & = & \textbf{0} \\
        f^+(\textbf{x}^*,\gamma^*) &=& \begin{pmatrix} \frac{\mu\delta\beta}{\mu+\delta\beta} \\ -\frac{\mu\delta\beta}{\mu+\delta\beta} \end{pmatrix} & \neq & 0 \\
    \end{array}
\end{equation}

Moreover

\begin{equation}
\label{eq:bifurcation_point_cond_2}
    H(\textbf{x}^*,\gamma^*) = 0
\end{equation}

$\frac{\partial f^-(\textbf{x}^*,\gamma^*)}{\partial \textbf{x}}$ e $\frac{\partial f^+(\textbf{x}^*,\gamma^*)}{\partial \textbf{x}}$ are invertible matrix, indeed

\begin{equation}
\label{eq:bifurcation_point_cond_3}
    \left| \frac{\partial f^-(\textbf{x}^*,\gamma^*)}{\partial \textbf{x}} \right| = \left| \frac{\partial f^+(\textbf{x}^*,\gamma^*)}{\partial \textbf{x}} \right| =  \frac{\mu^2\beta\delta}{\mu+\delta\beta} \neq 0
\end{equation}

Ine the end, knowing that 

\begin{equation}
    \begin{array}{ccc}
        \frac{\partial f^-(\textbf{x}^*,\gamma^*)}{\partial \textbf{x}} &=& \begin{pmatrix}
        -\mu-\beta\delta & -\frac{\mu\beta}{\mu+\beta\delta} \\
        \mu\delta & 0 \end{pmatrix} \\
        \frac{\partial f^+(\textbf{x}^*,\gamma^*)}{\partial \textbf{x}} &=& \begin{pmatrix}
        -\mu & 0 \\ 0 & -\frac{\mu\beta}{\mu+\delta\beta} \end{pmatrix} \\
        \frac{\partial f^-(\textbf{x}^*,\gamma^*)}{\partial\gamma} &=& \begin{pmatrix} 0 \\ -\delta \end{pmatrix} \\
        \frac{\partial f^+(\textbf{x}^*,\gamma^*)}{\partial\gamma} &=& \begin{pmatrix} 0 \\ -\delta \end{pmatrix} \\
        \frac{\partial H(\textbf{x}^*,\gamma^*)}{\partial \textbf{x}} &=& (0,1) \\
        \frac{\partial H(\textbf{x}^*,\gamma^*)}{\partial \gamma} &=& 0 \\
    \end{array}
\end{equation}

It happens that

\begin{equation}
\label{eq:bifurcation_point_cond_4}
    \begin{array}{ccccc}
        \frac{\partial H(\textbf{x}^*,\gamma^*)}{\partial \gamma} - \frac{\partial H(\textbf{x}^*,\gamma^*)}{\partial \textbf{x}} \left[\left(\frac{\partial f^-(\textbf{x}^*,\gamma^*)}{\partial\textbf{x}}\right)^{-1}\frac{\partial f^-(\textbf{x}^*,\gamma^*)}{\partial\gamma}
        \right] &=& -\frac{\left(\mu+\beta\delta\right)^2}{\beta\mu^2} & \neq & 0 \\
        \frac{\partial H(\textbf{x}^*,\gamma^*)}{\partial \gamma} - \frac{\partial H(\textbf{x}^*,\gamma^*)}{\partial \textbf{x}} \left[\left(\frac{\partial f^+(\textbf{x}^*,\gamma^*)}{\partial\textbf{x}}\right)^{-1}\frac{\partial f^+(\textbf{x}^*,\gamma^*)}{\partial\gamma}
        \right] &=& -\frac{\delta}{\mu} & \neq & 0 \\ 
    \end{array}
\end{equation}

Equations (\ref{eq:bifurcation_point_cond_1}), (\ref{eq:bifurcation_point_cond_2}), (\ref{eq:bifurcation_point_cond_3}) and (\ref{eq:bifurcation_point_cond_4}) fulfill condition to define $(x^*,\gamma^*)$ as a boundary equilibrium bifurcation point \cite[pp. 235]{bib:di_bernardo}.
\end{proof}