\begin{theorem}
$\gamma = \beta-\mu$ is a bifurcation point for system \ref{eq:sir_model_3}. Moreover, in the neighborhood of $(x_s,x_i) = (1,0)$ system is topologically equivalent to normal form
\begin{equation}
    \dot{\xi} = \nu\xi + \xi^2 + O\left(\|(\xi,\nu)\|^3\right)
\end{equation}
\end{theorem}

\begin{proof}

First of all can be useful do a change of variables

\begin{equation}
    \label{eq:chage_of_variable_1}
    \begin{array}{ccc}
    x &=& x_s - 1 \\
    \dot{x} &=& \dot{x_s} \\
    y &=& x_i \\
    \dot{y} &=& \dot{x_i} \\
    \alpha &=& \beta - \gamma - \mu
    \end{array}
\end{equation}

Substituting \ref{eq:chage_of_variable_1} in \ref{eq:sir_model_3} it happens that:

\begin{equation}
    \label{eq:sir_model_4}
    \begin{array}{ccc}
        \dot{x} &=& -\mu x -\beta y - \beta xy \\
        \dot{y} &=& (\alpha+\beta) y +\beta xy
    \end{array}
\end{equation}

Where bifurcation point became $(x,y,\alpha) = (0,0,0)$.

Eigenvalues of system \ref{eq:sir_model_4} at bifurcation point are

\begin{equation}
    \begin{array}{ccc}
        \lambda_1 &=& -\mu \\
        \lambda_2 &=& \alpha
    \end{array}
\end{equation}

Eigenvector associated at eigenvalue $\lambda_1$ is
\begin{equation}
    \begin{pmatrix}
        0 & \beta \\ 0 & -\mu-\alpha-\beta
    \end{pmatrix}
    \begin{pmatrix}
        \alpha_1 \\ \alpha_2
    \end{pmatrix} =
    \begin{pmatrix}
        0 \\ 0
    \end{pmatrix}
    \implies
    \begin{pmatrix}
        1 \\ 0
    \end{pmatrix}
\end{equation}

Eigenvector associated at eigenvalue $\lambda_2$ is
\begin{equation}
    \begin{pmatrix}
        \alpha+\mu & \beta \\ 0 & 0
    \end{pmatrix}
    \begin{pmatrix}
        \alpha_1 \\ \alpha_2
    \end{pmatrix} =
    \begin{pmatrix}
        0 \\ 0
    \end{pmatrix}
    \implies
    \begin{pmatrix}
        1 \\ -\frac{\alpha+\mu}{\beta}
    \end{pmatrix}
\end{equation}

So locally at bifurcation point happens

\begin{equation}
    \label{eq:uv}
    \begin{pmatrix} x \\ y \end{pmatrix}
    = u
    \begin{pmatrix} 1 \\ 0 \end{pmatrix} + v
    \begin{pmatrix} 1 \\ -\frac{\alpha + \mu}{\beta}\end{pmatrix}
    \implies \begin{pmatrix} x \\ y \end{pmatrix}
    = \begin{pmatrix} u + v \\ -\frac{\alpha + \mu}{\beta} v \end{pmatrix}
\end{equation}

Consequentialy
\begin{equation}
    \label{eq:uv_dot}
    \begin{pmatrix} \dot{x} \\ \dot{y} \end{pmatrix}
    = \begin{pmatrix} \dot{u} + \dot{v} \\ -\frac{\alpha + \mu}{\beta} \dot{v} \end{pmatrix}
\end{equation}

Substituting \ref{eq:uv} and \ref{eq:uv_dot} in \ref{eq:sir_model_4}, it happens

\begin{equation}
    \label{eq:sir_model_5}
    \begin{array}{ccc}
        \dot{u} &=& -\mu u + \left[ \beta(\alpha+\mu)-\beta \right]uv + \left[ \beta(\alpha + \mu) -\beta \right]v^2 \\
        \dot{v} &=& (\alpha + \beta u)v + \beta v^2
    \end{array}
\end{equation}

Given that eigeinvector associated with eigenvalue $\lambda_1$ is a stable manifold, while eigenvector associated with eigenvalue $\lambda_2$ is a center manifold, using Center Manifold Theorem \cite[pp. 303]{bib:khalil} locally at bifurcation point happens

\begin{equation}
    u = h(v,\alpha) = a+bv+c\alpha+dv^2+e\alpha^2+fv\alpha+O\left(\|(v,\alpha)\|^3\right)
\end{equation}

Moreover
\begin{equation}
    v = 0 \;\;\; \alpha = 0 \implies u = 0 \implies a = 0
\end{equation}

Then

\begin{equation}
    \label{eq:u_1}
    u = h(v,\alpha) = bv+c\alpha+dv^2+e\alpha^2+fv\alpha+O\left(\|(v,\alpha)\|^3\right)
\end{equation}

Now

\begin{equation}
    \label{eq:dotu_1}
    \dot{u} = \frac{\partial h}{\partial v}\dot{v} + \frac{\partial h}{\partial \alpha}\dot{\alpha} = \frac{\partial h}{\partial v}\dot{v} \;\;\;\; \text{in quanto} \;\;\;\; \dot{\alpha} = 0
\end{equation}

Remembering that \ref{eq:u_1}

\begin{equation}
    \begin{array}{ccc}
        \frac{\partial h}{\partial v} &=& b + 2dv+f\alpha+O\left(\|(v,\alpha)\|^3\right) \\
        \dot{v} &=& \alpha v + \beta v^2 + \beta b v^2 + \beta bc \alpha v + O\left(\|(v,\alpha)\|^3\right)
    \end{array}
\end{equation}

Then 
\begin{equation}
    \label{eq:dotu_2}
    \dot{u} = (b+\beta bc)\alpha v + (\beta b + \beta b^2) v^2 + O\left(\|(v,\alpha)\|^3\right)
\end{equation}

Substituiting \ref{eq:u_1} in \ref{eq:uv_dot} it happens that
\begin{equation}
    \label{eq_dotu_3}
    \dot{u} = -(\mu b)v -(\mu c)\alpha +(-\mu d + \beta\mu b -\beta b+\beta\mu-\beta)v^2-(\mu e)\alpha^2+(-\mu f+\beta\mu c -\beta c)\alpha v+O\left(\|(v,\alpha)\|^3\right)
\end{equation}

Comparing \ref{eq:dotu_2} with \ref{eq_dotu_3} and, solving the system, it happens that

\begin{equation}
    \label{eq:coefficients}
    \begin{array}{ccc}
        b &=& 0 \\
        c &=& 0 \\
        d &=& \beta-\frac{\beta}{\mu} \\
        e &=& 0 \\
        f &=& 0
    \end{array}
\end{equation}

Substituiting \ref{eq:coefficients} in \ref{eq:u_1} it happens
\begin{equation}
    u = \left(\beta - \frac{\beta}{\mu}\right) v^2 + O\left(\|(v,\alpha)\|^3\right)
\end{equation}

So equation associted to critic eigenvector
\begin{equation}
    \dot{v} = \alpha v + \beta v^2 + O\left(\|(v,\alpha)\|^3\right)
\end{equation}

Defining $\nu = \alpha$ and $v = \frac{\xi}{\beta}$ siit happens

\begin{equation}
    \dot{\xi} = \nu\xi + \xi^2 + O\left(\|(\xi,\nu)\|^3\right)
\end{equation}
\end{proof}