\begin{theorem}
Suppose that $  R_0 > 1+\delta\frac{\beta}{\mu}$. System \ref{eq:sir_model_pws_1} has a not endemic equilibrium point $x_{ne}^*=(1,0)$ (saddle point) and a endemic pseudo-equilibrim point $x_{e}^*=\left(1-\delta\left(\frac{\gamma+\mu}{\mu}\right), \delta\right)$ (locally asymptotically stable).
\end{theorem}

\begin{proof}
Lemma \ref{th:equilibria_lockdown} shows that $f^+$ has no one admissible equilibrium point.

Lemma \ref{th:sliding_equilibria} shows that $f_\Sigma$ has no one equilibrium point because by assumption $R_0 < 1 + \frac{\delta\beta}{\mu}$.

For vector field $f^-$, Theorem \ref{th:sir_equilibria} shows that system has following equilibrium point

\begin{equation}
    \begin{array}{ccc}
    x_{ne} &=& (1,0) \\
    x_{e} &=& \left(\frac{1}{R_0}, \frac{\mu}{\beta}\left(R_0 - 1\right)\right)
    \end{array}
\end{equation}

Given that for $f^-$ is mandatory fulfill condition $H(x) < 0$, this implies $x_i < \delta$. But for not endemic equilibrium point $x_{ne}$ happnes

\begin{equation}
    \delta > 0 = x_i
\end{equation}

So $x_{ne}$ is an equilibrium point for System (\ref{eq:sir_model_pws_1}) that, as has been shown in Theorem \ref{th:R0_major_then_1_Equilibria}, is saddle point.

FOr te endemic equilibrium point is mandatory to fulfill condition

\begin{equation}
    H(x) < 0 \implies \delta > x_i = \frac{\mu}{\beta}\left(R_0 - 1\right) > \frac{\mu}{\beta}\frac{\beta}{\mu}\delta = \delta
\end{equation}

But this is a contradiction, so endemic equilibrium point is not an admissible equilibrium point.

FOr the sliding surface, by assumption $ 1+\delta\frac{\beta}{\mu} > R_0 > 1$. So, because conditions of Lemma \ref{th:sliding_equilibria} are fullfilled, the sliding surface has an admissible pseudo-equilibrium point locally asymptotically stable at $x_{e}^*=(1-\delta\frac{\left(\gamma+\mu\right)}{\mu},\delta)$.
\end{proof}