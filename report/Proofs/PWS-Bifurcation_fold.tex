\begin{theorem} Given System (\ref{eq:sir_model_pws_2}), point $(x,y,\gamma) = (0,0,0) $ is a bifurcation point of type persistence.
\end{theorem}

\begin{proof}
At bifurcation point $(x^*,y^*,\alpha^*) = (0,0,0) $ it happens that

\begin{equation}
    \begin{array}{ccccc}
        N &=& \frac{\partial f^-(0,0,0)}{\partial \textbf{x}} &=& 
        \left(
            \begin{array}{cc}
                -\mu-\beta\delta & \frac{\beta\mu}{\mu+\delta\beta} \\
                \beta\delta & 0 \\
            \end{array}
        \right) \\
        C^T &=& \frac{\partial H(0,0,0)}{\partial \textbf{x}} &=& \left(0,1\right) \\
        D &=& \frac{\partial H(0,0,0)}{\partial\alpha} &=& 0 \\
        M &=& \frac{\partial f^-(0,0,0)}{\partial\alpha} &=& \left( \begin{array}{c} 0 \\ -\delta \end{array} \right) \\
        E &=& f^+(0,0,0)-f^-(0,0,0) &=& \left( \begin{array}{c} \frac{\beta\delta\mu}{\mu+\delta\beta} \\ -\frac{\beta\delta\mu}{\mu+\delta\beta} \end{array} \right) \\
    \end{array}
\end{equation}

Now,
\begin{equation}
\label{eq:bifurcation_theorem_cond_1}
    det(N) = -\frac{\beta^2\delta\mu}{\mu+\delta\beta}
\end{equation}

\begin{equation}
\label{eq:bifurcation_theorem_cond_2}
    D-C^TN^{-1}M = -\frac{(\mu+\mu\delta)^2}{\beta^2\mu} \neq 0
\end{equation}

\begin{equation}
\label{eq:bifurcation_theorem_cond_3}
    C^TN^{-1}E = -\frac{\mu + \beta\delta}{\beta} \neq 0
\end{equation}

\begin{equation}
\label{eq:bifurcation_theorem_cond_4}
    C^TN^{-1}E = -\frac{\mu + \beta\delta}{\beta} < 0
\end{equation}

Equation (\ref{eq:bifurcation_theorem_cond_1}), (\ref{eq:bifurcation_theorem_cond_2}), (\ref{eq:bifurcation_theorem_cond_3}) and (\ref{eq:bifurcation_theorem_cond_4}) fulfill all condition to say that $(x,y,\gamma) = (0,0,0) $ is a bifurcation point of type persistence \cite[pp. 236]{bib:di_bernardo}.
\end{proof}