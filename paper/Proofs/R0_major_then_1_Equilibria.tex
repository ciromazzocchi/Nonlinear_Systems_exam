\begin{theorem}
\label{th:R0_major_then_1_Equilibria}
Given $R_0 > 1$, then system \ref{eq:sir_model_3} has two equilibrium point that are not endemic equilibrium point (saddle point) and endemic equilibrium point (locally asymtotically stable).
\end{theorem}

\begin{proof}
Considering not endemic equilibrium point. It obvious verify that $x_{ne}^* \in \Omega$.

Given \ref{eq:eigenvalues_ne} and remembering that by assumption $\mu > 0$ and $\gamma > 0$, so

\begin{equation}
    \lambda_1 = -\mu < 0
\end{equation}

\begin{equation}
    \lambda_2 = \beta - \gamma - \mu = (\gamma + \mu)\left(\frac{\beta}{\gamma + \mu} - 1\right) = (\gamma + \mu)(R_0 - 1) > 0
\end{equation}

Then not endemic equilibrium point has a real positive eigenvalue and a real negative eigenvalue, so is a saddle point.

Now considering the endemic equilibrium point. First of all it is mandatory verify that this point is a part of domain $\Omega$.

\begin{equation}
    x_s^* = \frac{1}{R_0}
\end{equation}

But, by assumption

\begin{equation}
    R_0 > 1 \implies x_s^* < 1
\end{equation}

Moreover, 

\begin{equation}
    x_s^* < 0 \implies R_0 < 0  
\end{equation}

But this cannot be true because $R_0 > 1$ by assumption, so

\begin{equation}
    0 < x_s^* < 1
\end{equation}

For $x_i^*$

\begin{equation}
    1 < R_0 = \frac{\beta}{\gamma + \mu} \implies \frac{1}{\beta} < \frac{1}{\gamma + \mu} \implies 0 < \frac{1}{\gamma + \mu} - \frac{1}{\beta} \implies 0 < \frac{\mu}{\gamma + \mu} - \frac{\mu}{\beta} = x_i^*
\end{equation}

Moreover

\begin{equation}
    1 < R_0 = \frac{\beta}{\gamma + \mu} \implies \beta > \gamma + \mu > \mu \implies x_i^* = \frac{\mu}{\gamma + \mu} - \frac{\mu}{\beta} < 1 - \frac{\mu}{\beta} < 1
\end{equation}

In the end

\begin{equation}
    x_s^* + x_i^* = \frac{1}{R_0} + \frac{\mu}{\beta}\left(R_0 - 1\right) < \frac{1}{R_0} + \frac{\mu+\gamma}{\beta}\left(R_0 - 1\right) =  \frac{1}{R_0} + \frac{1}{R_0}\left(R_0 - 1\right) = 1
\end{equation}

So

\begin{equation}
    0 < x_s^* < 1, \;\; 0 < x_i^* < 1, \;\; x_s^* + x_i^* < 1 \implies x_e^* \in \Omega 
\end{equation}

Now it need to analyze eigenvalues. Substituiting \ref{eq:r0_definition} in \ref{eq:eigenvalues_e} it happens:

\begin{equation}
    \lambda_{1,2} = -\frac{\mu R_0}{2} \pm \frac{1}{2}\sqrt{\left(\mu R_0\right)^2-\frac{4\mu}{\gamma + \mu}(R_0 - 1)}
\end{equation}

Now, it happnes that

\begin{equation}
    \left(\mu R_0\right)^2-\frac{4\mu}{\gamma + \mu}(R_0 - 1) = \left(\mu R_0\right)^2-\frac{4\mu R_0}{\beta}(R_0 - 1)
\end{equation}

Noting that

\begin{equation}
    R_0 > 1 \implies -\frac{4\mu R_0}{\beta}(R_0 - 1) < 0
\end{equation}

So it can be three possible scenarios:

\begin{itemize}
    \item $1 < R_0 < \frac{4}{4-\beta \mu} \implies \lambda_{1,2}$ are complex conjugate eigenvalues with negative real part (stable focus)
    
    \item $R_0 = \frac{4}{4-\beta \mu} \implies \lambda_{1,2}$ are coincident negative real eigenvalues (attractive star)
    
    \item $R_0 > \frac{4}{4-\beta \mu} \implies \lambda_{1,2}$ are distinct negative real part ( node )
\end{itemize}

In all three scenarios eingevalues have negative real part, so equilibrium point is an hyperbolic equilibrium point locally asymptotically stable.
\end{proof}