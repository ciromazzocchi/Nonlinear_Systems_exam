\begin{lemma}
Given the system \ref{eq:sir_model_pws_1}, region $\Sigma$ is an attractive surface for $\frac{1}{R_0} < x_s \leq 1$.
\end{lemma}

\begin{proof}
By definition, $\Sigma$ is an attractive surface if $f^+_N < 0$ and $f^-_N > 0$.

From \ref{eq:f+}, solving equation and remembering that $x_i = \delta$, it happens that

\begin{equation}
    f^+_N = -(\gamma + \mu)\delta
\end{equation}

By assumption:
\begin{itemize}
    \item $\delta > 0$
    \item $\gamma + \mu > 0$
\end{itemize}
So $f^+_N < 0 \;\;\;\;\; \forall x_s \in \Omega$

From \ref{eq:f-}, solving equation and remembering that $x_i = \delta$, it happens that

\begin{equation}
    f^-_N = \delta\beta\left(x_s-\frac{1}{R_0}\right)
\end{equation}

By assumption:
\begin{itemize}
    \item $\delta > 0$
    \item $\beta > 0$
\end{itemize}
So $\frac{1}{R_0} < x_s \leq 1 \implies f^-_N > 0$

As consequence $\Sigma$ is an attraction surface for $\frac{1}{R_0} < x_s \leq 1$ 
\end{proof}